\documentclass[12pt, oneside]{article}   	% use "amsart" instead of "article" for AMSLaTeX format
\usepackage{color}
\usepackage{geometry}                		% See geometry.pdf to learn the layout options. There are lots.
\geometry{letterpaper}                   		% ... or a4paper or a5paper or ... 
%\geometry{landscape}                		% Activate for for rotated page geometry
%\usepackage[parfill]{parskip}    		% Activate to begin paragraphs with an empty line rather than an indent
\usepackage{graphicx}				% Use pdf, png, jpg, or eps§ with pdflatex; use eps in DVI mode
								% TeX will automatically convert eps --> pdf in pdflatex		
\usepackage{amssymb}
\usepackage{amsmath}
\usepackage[compact]{titlesec}
\usepackage{float}
\usepackage{pdflscape}
\usepackage{rotating}
\usepackage{soul}
\usepackage{longtable}
\usepackage{threeparttable}
\usepackage{lineno}
\usepackage[round]{natbib} %round makes parentheses instead of square brackets
\usepackage{url}
%\usepackage{authblk}
\setcounter{secnumdepth}{4}
\titleformat{\paragraph}
{\normalfont\normalsize\bfseries}{\theparagraph}{1em}{}
\titlespacing*{\paragraph}
{0pt}{3.25ex plus 1ex minus .2ex}{1.5ex plus .2ex}
\graphicspath{ {images/} }

\title{Notes on clownfish persistence}

\begin{document}
%\linenumbers
\date{\today}
\maketitle{}
\section*{Literature notes}

\subsection*{Clownfish life history}
\paragraph*{\textit{Amphiprion clarkii}}
\textbf{\underline{Rank, reproduction, and breeding behavior}}
\begin{itemize}
\item \cite{hattori1991life}: different life history pathways found in population of \textit{A. clarkii} in Japan in area where they are the only anemonefish: 1) subadult male $\rightarrow$ subadult female $\rightarrow$ adult female, 2) subadult male $\rightarrow$ adult male $\rightarrow$ adult female, 3) subadult male $\rightarrow$ adult male; 6 color phases seen - three for nonbreeders and three for breeders: small nonbreeders = transparent caudal fins, larger nonbreeders = orange borders on caudal fins or creamy white fins, male breeders =  orange caudal fins, female breeders =  white caudal fins, size=canging fish = faintly orange caudal fins basally, whitish distally; p.152: "the timing of femininity differentiation [which determines which life history path a fish takes] is not genetically determined but socially regulated [see citations for info on sex change of tropical anemonefishes]." "... femininity differentiation in the nonbreeder state is common and most females are recruited directly from subadults [as opposed to functioning as males first and then switching sexes]"
\item \cite{hattori1991life}: females produced 1-9 clutches in one breeding season (temperate waters off Japan, clear breeding season and non-breeding season)
\item \cite{hattori1991life}: p.153: \textit{A. clarkii} lives in both temperate and tropical waters, might be less dependent on host sea anemones and a "more efficient swimmer than other anemonefishes" (cites Allen 1972). "This may mean that this species has no typical social unit even in tropical waters." See some possible evidence of lack of a typical social unit with nonbreeders living outside of breeding pairs' territories in tropical reefs off of Ryukyu Islands in Japan (Hattori unpublished)
\item \cite{ochi1989mating}: p.258: could be differences in the mating system/behavior of clownfish depending on how dense the host anemones are in the habitat and how costly it is to move between them; \textit{A. clarkii} in temperate areas off coast of Japan "showed many differences in behavior and morphology in comparison with conspecifics and othre anemonefishes from the primary habitats (coral reefs)."
\item \cite{ochi1989mating}: female body size correlated with total eggs laid in clutches only in some years (yes in 1983, 1986, 1987; no in 1984, 1985) (p.261). Mean annual clutch size "estimated at about 137000 eggs"; mean egg density 79 eggs per $\text{cm}^2$, females with mates throughout whole breeding season laid 4-9 clutches (mean=5.9, SD=1.3), largest clutches estimated to have about 1600-5400 eggs (mean=3000) based on egg density (p.261); body size of females correlated positively with mean size of clutch
\item \cite{ochi1989mating}: breeding pairs separate both during the breeding season and during the nonbreeding season, for 4 main reasons: 1) one dies or disappears, 2) one moves to another territory, 3) one is expelled from territory by an invader, 4) one mate expels the other from the territory (p.263). Next several pages have lots of details on what happened in observations of these cases, characteristics of new mates, etc.
\item \cite{ochi1989mating}: p.273: "Although \textit{A. clarkii} basically kept a pair bond with one mate, they seem to have a tendancy to acquire additional mates if an opportunity arises."
\item \cite{ochi1989mating}: see paternal egg care - might make it hard for a male to control/keep multiple mates because defending clutches and territory takes time (p.273)
\item \cite{ochi1989mating}: in this study, not see evidence for Moyer 1980 suggestion that males change sex to escape egg-care duties and dominance by females (to increase LEP), didn't see cases of a male leaving his mate to change sexes, many males who lost their mates didn't change to female and instead paired with other females, possibly because the time it takes to transition to female is lost breeding time if instead they could just mate with other females (discussed p.273)
\item \cite{moyer1986longevity}: the individual observed for 11 years and thought to live to at least 13 was "estimated to have contributed about 160,000 propagules in its lifetime," spent 3 years as a functional male, then outlived 3 mates as a female; fertilized about 45,000 eggs during 3 years as a male, then spawned about 115,000 eggs as a female; fecundity estimate at this site (Miyake-jima in Japan) is 17,500 eggs/yr/female (from \cite{bell1976notes})
\item \cite{moyer1986longevity}: why are these species protandrous? Turns out they are fairly mobile and "individuals could survive for relatively long periods of time without an anemone shelter", at least at Miyake-jima site in Japan and another site (p.138: "Y. Hirose (pers comm.) found \textit{A. clarkii} to be highly mobile in relatively dense populations at Sesoko Island, Okinawa.", so high fecundity of females and obligate anemone-dwelling might not be a good explanation. Maybe life expectancy is greater for females; "These data suggest that males are possibly more susceptible to displacement and predation than females, perhaps due to the fact that care of nest and eggs is primarily a male role in \textit{Amphiprion}," makes feeding selectivly and regularly harder, higher vulnerability to predation due to "conspicious aggression related to nest defense", and harder to fight off similar-sized males that don't have to spend energy on nest care
\end{itemize}

\textbf{\underline{Home range}}
\begin{itemize}
\item \cite{hattori1991life}: at Murote Beach on west coast of Shikoku Island, Japan (where only anemonefish species present is \textit{A. clarkii}, "pairs of breeders held territories including one to ten sea anemones," where the average density of sea anemones was 5.7 anemones/$100\text{m}^2$; of all the anemones in 2 areas, 58.5\% and 85.9\% were "controlled by breeding pairs." "Nonbreeders had their home ranges in the outskirts of and interstices between territories of breeding pairs." (p.141) (so the non-breeders weren't in the same groups living on the anemones with breeders, just at lower rank?). Average size of anemones smaller in the territories of non-breeders than of breeders. p.153: "In Murote Beach, however, host sea anemones usually occur in dense populations compared to tropical waters (see some citations). Under such conditions, \textit{A. clarkii} moves between hosts and has no social unit typical of the tropical anemonefishes (citations). Subadults have home ranges outside of breeding pairs' territories."
\item \cite{ochi1989mating}: p.257: "The general social unit of the anemonefishes is an isolated group consisting of a monogamous breeding pair and a varying number of nonbreeders. In the present study area [temperate waters off Japan], however, monogamous pairs established territories almost contiguous to otres and nonbreeders had home ranges on the outskirts of the pairs' territories. The high host population density allows \textit{A. clarkii} to move between hosts for searching for mates and acquiring additional mates."
\item \cite{ochi1989mating}: p.259: [in temperate waters off Japan, Uwa Sea, west of Shikoku Island, with high host density], home ranges of adults of same sex rarely overlap, mean area of a home range that includes both the ranges of a male and a female pair was about $33.9\text{m}^2$ with SD of $10.3\text{m}^2$.
\end{itemize}

\textbf{\underline{Movement}}
\begin{itemize}
\item \cite{ochi1989mating}: p.260: recruits tend to stay near host anemone until reach about 30mm in SL (cites Ochi 1986)
\end{itemize}

\textbf{\underline{Survival, mortality, lifespan}}
\begin{itemize}
\item \cite{moyer1986longevity}: one study site ("warm temperate waters of Miyake-jima, one of the Izu Islands of southern Japan ($34^{\text{o}}05$'N, $139^{\text{o}}30$'E)"), 14 years: observed a particular \textit{A. clarkii} over 11-year period, fish thought to be at least 13 when disappeared (identified the indidvidual by unusual color markings), doesn't think 13yr longevity is unusual for \textit{A. clarkii} at this site
\end{itemize}

\paragraph*{Other clownfish species}
\textbf{\underline{Rank, reproduction, and breeding behavior}}
\begin{itemize}
\item \cite{saenz2011connectivity}: \textit{Amphiprion polymnus} p.2956: "Anemonefish are considered monogamous with only the two biggest fish (breeders) been reproductively active in the fish colony [29]." Their parentage analysis date supports this. Citation 29 is Fautin and Allen 1992: Field guide to anemonefishes and their host sea anemones
\item \cite{buston2003social}: \textit{Amphiprion percula} subordinates adjust size and growth rate, "maintaining a well-defined size difference with respect to individuals above them in social rank," likely so that individuals that outrank them don't perceive them as a threat to their dominance and evict or kill them.
\item Settling clownfish can be "driven from the anemone within hours of their arrival [38,47]" - p.1884 in \cite{buston2011probability}, citations are \cite{buston2003forcible} and \cite{elliott1995host}
\item \cite{ochi1989mating}: p.258: could be differences in the mating system/behavior of clownfish depending on how dense the host anemones are in the habitat and how costly it is to move between them; Fricke 1979 found "no essential difference in the mating system between populations of \textit{A. akallopisos} in low and high host density"
\item \cite{sato2017marine} (study \textit{Amphiprion frenatus} and \textit{A. perideraion} in the Philippines (Mindanao): suggest that predation might not affect egg mortality much - p.10: "Anemonefish are generally protected by sea anemones, and the males care for the eggs until hatching (Buston \& Elith, 2011; Mariscal, 1970); therefore, the prescence of predators may not strongly affect reproduction and/or egg survival."
\end{itemize}

\textbf{\underline{Home range}}
\begin{itemize}
\item \cite{buston2011probability}: in 1km square area of \textit{Amphiprion percula} around Kimbe Island in Papua New Guinea, all 275 anemones were occupied by \textit{A. percula} (in Dec. 2004 when surveyed/sampled), which is "consistent with observations at other sites in Papua New Guinea [38,40]."
\end{itemize}

\textbf{\underline{Habitat}}
\begin{itemize}
\item \cite{salles2016genetic}: \textit{Amphiprion percula} on Kimbe Island: anemone species and location affects clownfish life history traits, like female size, number of eggs per clutch, juvenile growth rate, group size on the anemone, but not proxy for longevity. The species that had clownfish with larger females and more eggs per clutch produced a lower percentage of the local recruits, however.
\end{itemize}

\textbf{\underline{Survival, mortality, lifespan}}

\subsection*{Larval dispersal and connectivity}
\begin{itemize}
\item \citep{almany2017larval}: use parentage analysis to estimate connectivity and dispersal kernels for two different coral reef fish in the Kimbe Bay area, including the clownfish species \textit{Amphiprion percula}. Should review methods of kernel and connectivity estimates more closely. Talk about both local retention and self-recruitment, use a slightly different def of LR ($\frac{\text{\# larvae returning home}}{\text{\# larvae that left and settled somewhere}}$) than Burgess and Botsford usually use, need to think more about if/how that matters.
\item \citep{saenz2011connectivity}: different definitions of connectivity: genetic connectivity vs. demographicly connected populations; p.2954: "Coupled biophysical models have suggested that ecologically relevant larval dispersal in reef fishes occurs over scales of 10-100km in the Caribbean Sea and along the Great Barrier Reef" (see paper for specific citations); p.2955 "Empirical studies of demographic connectivity have suggested that variation in dispersal distance among species is more likely to be influence by geographical isolation and spacing of reefs than individual species characteristics."
\end{itemize}

\subsection*{Connectivity and dispersal}
\paragraph*{Connectivity terms and definitions}
\begin{itemize}
\item \textbf{local recruitment (local recruits)}: recruits from the natal local region (say within a metapopulation) but not the natal site \citep{saenz2011connectivity}
\item \textbf{self recruitment (self recruits)}: percentage of recruits to a site that originated from that site \citep{saenz2011connectivity}
\item \textbf{local connectivity}: the number of settlers at a given site not originated from that site but from another site within the local region (as opposed to immigrants from farther populations) \citep{saenz2011connectivity}
\end{itemize}
\paragraph*{Larval dispersal}
\begin{itemize}
\item \cite{buston2011probability}: observed dispersal distribution: distribution of distances between recruits and their parents (using anemones as the origin and destination locations); expected dispersal distance distribution: "distribution of distances between each and every anemone in the metapopulation" - compare these statistically for \textit{Amphiprion percula} around Kimbe Island; expected and observed dispersal distributions were different, probability of dispersing between two populations (anemones) declined with distance, larvae 5x as likely to successfully disperse 1m as 1km; p.1886: "This [the 5x more likely to disperse 1m than 1km] suggests that the \textit{A. percula} dispersal kernel is a unimodal leptokurtic distribution with a peak close to source, analogous to the majority of terrestrial seed dispersal kernels [6]."
\item \cite{daloia_self-recruitment_2013}: test a "simple exponential decay function" to see which one best captures pattern of self-recruitment at their focal site (Curlew); estimate the relative probability of dispersal ($y$) from any patch in the metapopulation ($j$) to the focal patch as $y_j = y_0b^{x_j}$ where $y_0=1$, $x_j$=distance to the focal site from patch $j$, b=decay rate (try a range of values from 0 to 1, find that b=0.915 fits data best); approach assumes "all reef patchs are occupied, population density and reproductive output are equal among patches, and larval dispersal is represented by the same function from each patch" (p.2568); estimate the predicted self-recruitment ($SR_p$) from the expontential decay function as: $SR_p = \frac{y_0}{\sum_{j=0}^{n} y_j}$, with n=224 reef patches, assume the exponential decay equation that fits the SR data best is the "putative dispersal kernel"
\item \cite{daloia_self-recruitment_2013}: p.2568: "Interestingly, one parent produced two of the self-recruiting offpsring. There were also two self-recruiting offspring that settled on the same sponge." $\rightarrow$ anecdotal evidence for packet dispersal?
\item \cite{daloia_self-recruitment_2013}: p.2570: "To advance our understanding of marine population connectivity, the focus must progress from making point estimates to fitting dispersal kernels to empirical data and testing the predictions derived from such kernels." Take a first step w/this paper, etc. ... p.2571: "The simple exponential kernel proposed in this study, and the probability distributions described above, represents phenomenological models of dispersal (Nathan \& Muller-Landau 2000). Phenomenological models characterize the functoin that best describes the observed relationship between dispersal and distance. These models are an important first step in quantifying dispersal patterns..."
\item \cite{d2015patterns}: study gobi on Belizian reefs (\textit{Elacatinus lori}, mean dispersal distance 1.7km, don't find any dispersal events $>$16.4km; define an empirical dispersal kernel as "a probability density function (p.d.f.) that can be integrated to yield the probability of successful dispersal over a given distance"; p.13941: "despite having an average 26-d larval phase and therefore the potential to disperse far via ocean currents, \textit{E. lori} exhibits a spatially restricted leptokurtic pattern of dispersal."
\item \cite{d2015patterns}: PLD for \textit{Elacatinus lori} about 26 days, which is about the median for reef fishes (cite Brothers \& Thresher 1985)
\item \cite{d2015patterns}: fit different functional forms to their data to estimate dispersal kernel, find exponential form fit best (p.13941): $f(x) = \lambda e^{-\lambda x}$, where $f(x) = \text{probability density}$, $x = \text{distance in km}$, $\lambda = \text{decay parameter} = 0.36$; found an effect of spatial variation of magnitude of $\lambda$ but not direction, settler length (proxy for age), PLD (counted using rings in the otoliths of settlers)
\item \cite{d2015patterns}: using a generalized linear model to evaluate whether or not (0 or 1) the possible dispersal pathways were use, found that distance, parent region, and sampling effort are related to probability of successful dispersal
\item \cite{sato2017marine}: looked at two clownfish species in an MPA between two fished zones (on Mindanao!), found lower density of anemones in MPA, higher abundance of predators in the MPA, lower abundance of clownfish per anemone in the MPA, and lower number of fish recruiting per anemone from outside areas and fished zones to MPA than to fished zones - they speculate this could be because of stronger top-down control/ lower survival in the MPA b/c of predator release from fishing pressure. They suggest there might be higher densities of anemones in the fished areas b/c of reduced coral cover and competition with coral. Seem like interesting/plausible ideas but not sure I'm totally conviced that they can attribute what they find here to trophic effects rather than just site effects (particulalry since they only look at one MPA and it is in the middle of the two fished sites - seems like just distance might account for fewer fish from unstudied areas settling there.) They use COLONY, give the assumptions/settings they use.
\end{itemize}
\paragraph*{Connectivity more generally}
\begin{itemize}
\item \cite{d2015patterns}: summary from their significance paragraph (p.13940): "Remarkably, the distance and individual travels is unrelated to the number of days it spends in the larval phase. These results suggest that simple distance-based models may be useful conservation tools and that MPAs that are close in space will accomodate short-distance dispersers."; p.13943 "Our findings demonstrate that, for \textit{E. lori}, the probability of succesful dispersal declines exponentially and predictably as a function of distance from source.", "Furthermore, based on observation of spatial genetic structure, we conclude that for some marine fish populations, there is strong congruence between the scale of demographic and genetic connectivity."
\item \cite{gaggiotti2017metapopulations}: talks about how metapopulations and measurement of connectivity differs between terrestrial and marine systems, can be more challenging in marine systems for many reason; proposes using a Bayesian hierarchical framework to incorporate different kinds of data to estimate connectivity in marine systems. The basic idea is to use biophysical modelling data as a prior and genetic and microchemistry raw data in the likelihood function to estimate "parameters of a probabilistic model of migration (connectivity) between sources and recruit sampling sites." (p.106)
\end{itemize}

\subsection*{Mark-recapture}
\paragraph{List of models to consider:}
\begin{itemize}
	\item survival constant, recapture constant
	\item survival constant, recapture time
	\item survival time, recapture constant
	\item survival constant, recapture distance
	\item survival size, recapture size
	\item survival size, recapture distance
	\item survival size+stage, recapture constant
	\item survival size+stage, recapture size+stage
	\item survival size+stage, recapture size
	\item survival size+stage, recapture distance
	\item survival size+stage, recapture stage
	\item survival stage, recapture constant
	\item survival stage, recapture distance
	\item survival size+color, recapture constant
	\item survival size+color, recapture distance
	\item survival color, recapture constant
	\item survival color, recapture distance
\end{itemize}

\subsection*{Persistence}
\begin{itemize}
\item \citep{hastings_persistence_2006}: p.6068: "These persistence conditions [that they present] allow us to answer a variety of essential questions. What is the persistence condition if habitats are heterogeneous with different per capita propagule production or survival and nonuniform patterns of dispersal as would arise from physical advective processes such as in wind or water? In a heterogenous system, what parts of the system would be most important to protect to achieve persistence? How does a network of reserves function to ensure persistence of a species when a single reserve cannot?"
\end{itemize}

\subsection*{Metapopulation framing}
\paragraph*{Metapopulations with extinction/colonization of patches}
\begin{itemize}
\item \cite{buston2011probability} refers to \textit{Amphiprion percula} as small breeding groups on sea anemones that can be considered populations within a metapopulation - here, sounds like he is classifying the fish on each anemone as a separate population. p.1883: "\textit{Amphiprion percula} live in small breeding groups that inhabit sea anemones. These groups can be considered populations within a metapopulation (\textit{sensu} [42]).", where 42 is the Hanski 1999 citation.
\item \cite{gaggiotti2017metapopulations}: p.98: "Here, however, we are in a completely different realm with metapopulations of marine species representing a counterpoint to the Glanville fritillary metapopulations. Indeed, most marine species spend part of their life in the plankton at the mercy of strong oceanic currents and were initially seen as living in demographically open populations (Caley et al. 1996). On the other hand, butterflies were initially seen as living in closed populations because of their limited dispersal abilities and the patchy distribution of their host plants (Ehrlich et al. 1975). In both cases, however, the truch lies somewhere between these two extremes (cf. Hanski 1999, Hellberg 2009)."
\end{itemize}

\paragraph*{Metapopulations without extinction/colonization of patches}

\subsection*{Connectivity in non-marine systems}

\subsection*{Population parameters from the literature}
Notes on how life history and demographic parameters were estimated or calculated in other similar studies, numbers and sources in table below:
\begin{itemize}
\item Mortality: directly estimated from genetic recaptures \citep{salles_coral_2015}
\item Juveniles carrying capacity: highest value of mean number of juveniles per anemone of 4 sampling periods multiplied by the number of anemones in the subpopulation \citep{salles_coral_2015}
\item Adult carrying capacity: number of anemones in the subpopulation (b/c can only have one breeding pair per anemone) \citep{salles_coral_2015}
\end{itemize}

\begin{longtable} { |p{1.5in}|p{1.0in}|p{1.5in}|p{1.5in}| } 
\hline{}
\textbf{Parameter} & \textbf{Value} & \textbf{Source} & \textbf{Notes} \\ \hline
Mortality for various stages, male and female & juveniles: 0.18-0.49, males: 0.09-0.44, females: 0.19-0.55 & \citep{salles_coral_2015} & for \textit{Amphiprion percula} population around Kimbe Island in Kimbe Bay, see Table 1 in their paper for more details \\ \hline
Percent self-recruits for whole Kimbe Bay island pop over multiple years of study & 20.9\% & \citep{salles_coral_2015} & 5 sampling events from 2005-2013, 1067 total recruits sampled \\ \hline
Percent local recruits for whole Kimbe Bay island pop over multiple years of study & 33.3\% & \citep{salles_coral_2015} & 5 sampling events from 2005-2013, 1067 total recruits, local recruits means recruits from the other Kimbe Bay island subpopulations \\ \hline
Percent immigrant recruits for whole Kimbe Bay island pop over multiple years of study & 45.8\%  & \citep{salles_coral_2015} & 5 sampling events from 2005-2013, 1067 total recruits, immigrants recruits means recruits not from any of the Kimbe Bay Island subpopulations \\ \hline
\end{longtable}

\subsection*{To-reads}
\paragraph*{To read now}
\begin{itemize}
\item Larval dispersal: \cite{buston2013marine}
\item Metapopulations: review \cite{hanski1998metapopulation}, Hanski and Gaggiotti article (and probably others) in their 2004 metapopulation book, \cite{puckett2016metapopulation}
\item Connectivity: \cite{almany2007local}, \cite{christie2010larval}, \cite{cowen_scaling_2006}, review \cite{jones2009larval}, \cite{selkoe2011marine}
\item Genetic estimates of dispersal/ parentage analysis: \cite{broquet2009molecular}, \cite{jones2010practical}, \cite{manel2003landscape}, \cite{christie2017disentangling}
\item For deriving a new metric for local contribution to persistence: \cite{botsford_connectivity_2009}, \cite{botsford2009sustainability}, \cite{white_population_2010}
\end{itemize}
\paragraph*{Possible sources to check out in the future}
\begin{itemize}
	\item Anemones as habitat: rapid growth, high mortality (\cite{o2018giant})
\item Connectivity: 
\item Connectivity in non-marine systems: \cite{nathan2000spatial}, \cite{jordano2007differential}, \cite{nathan2008mechanisms}
\item Estimating dispersal: \cite{taylor2000evaluating}
\item Clownfish (\textit{Amphiprion clarkii}): \cite{srinivasan1999experimental}, \cite{bell1976notes}, \cite{moyer1980influence}, \cite{fricke1983social}, \cite{fricke1977monogamy}, \cite{moyer1976geographical}, \cite{moyer1976reproductive}, \cite{ochi1985temporal}, \cite{yanagisawa1986step} 
\item \textit{A. clarkii} studies:
\item Studies on other clownfish species: \cite{moyer1978protandrous} (protandrous hermaphrotidism), \cite{moyer1973territorial} (territorial behavior), \cite{ross1978reproductive} (reproductive behavior)
\item Anemonefish more generally: \cite{fautin1986anemonefishes}
\item Reef fish dispersal and PLD more generally: \cite{brothers1985pelagic}
\item Mating behavior and sex-changing in reef fish: \cite{warner1984mating}
\item Adaptive sequential hermaphroditism: \cite{warner1975adaptive}
\item How larvae might be able to navigate/behavior to affect dispersal: \cite{leis2011nemo}
\item Genetic structure of populations, using that to estimate connectivity: \cite{selkoe2014emergent}, \cite{selkoe2010taking}, \cite{selkoe2006current}
\item Metapopulations: \cite{gaggiotti2004combining} (might have a Bayesian framework for combining data to look at colonization possibilities)
\end{itemize}

\section*{Leyte clownfish notes}

\section*{Methods and analysis notes}
\subsection*{Estimates of survival and recapture probability - mark-recapture work}

\section*{Questions}
\subsection*{Brainstorms [to be classified below]}
\paragraph*{Can we answer with present data?}
\paragraph*{Can we answer with future data collection?}
\paragraph*{Unsure how or hard to answer}

\section*{Work plan and brainstorms}
\begin{itemize}
\item Bigger picture questions: 
\end{itemize}

\section*{Analysis ideas progress}
\begin{itemize}
\item Possible metric idea: $\frac{\text{number settlers returning that are from that site}}{\text{number settlers the site needs to receive on average to stay persistent}}$ If assume that larval mortality is same for larvae released from or arriving to all sites, might be able to get around trying to actually estimate it? Need to think about this a bit more... Then would just need to calculate survival from settler stage to some average adult lifespan stage to calculate how many settlers need to arrive per adult for replacement. Still need to think this through a bit, should play around to see what kinds of data, survival estimates, etc. would be necessary, see if those rates are estimatable from the database data...
\end{itemize}

\section*{Questions}

\bibliography{../../../BibTexReferences}
\bibliographystyle{plainnat}
%\bibliography{../../../BibTexReferences}
%\bibliographystyle{plainnat}

\end{document}