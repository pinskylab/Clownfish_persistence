\documentclass[12pt, oneside]{article}   	% use "amsart" instead of "article" for AMSLaTeX format
\usepackage{color}
\usepackage{geometry}                		% See geometry.pdf to learn the layout options. There are lots.
\geometry{letterpaper}                   		% ... or a4paper or a5paper or ... 
%\geometry{landscape}                		% Activate for for rotated page geometry
%\usepackage[parfill]{parskip}    		% Activate to begin paragraphs with an empty line rather than an indent
\usepackage{graphicx}				% Use pdf, png, jpg, or eps§ with pdflatex; use eps in DVI mode
								% TeX will automatically convert eps --> pdf in pdflatex		
\usepackage{amssymb}
\usepackage{amsmath}
\usepackage[compact]{titlesec}
\usepackage{float}
\usepackage{pdflscape}
%\usepackage{rotating}
\usepackage{soul}
%\usepackage{longtable}
%\usepackage{threeparttable}
%\usepackage{lineno}
\usepackage[round]{natbib} %round makes parentheses instead of square brackets
\usepackage{url}
%\usepackage{authblk}
\setcounter{secnumdepth}{4}
\titleformat{\paragraph}
{\normalfont\normalsize\bfseries}{\theparagraph}{1em}{}
\titlespacing*{\paragraph}
{0pt}{3.25ex plus 1ex minus .2ex}{1.5ex plus .2ex}
\graphicspath{ {images/} }

\title{Mark-recapture}

\begin{document}
%\linenumbers
\date{}
\maketitle{}

\section*{Growth}
We used the R package fishmethods \citep{nelson2018fishmethods}, which uses the methods of \cite{hampton1991estimation} to estimate von Bertalanffy growth curves. Using our mark-recapture data, including both genetic and tagged recaptures, we broke each recapture history into recapture pairs (so a fish caught 4 times has 3 pairs of captures).

Notes from reading \citep{hampton1991estimation}:
\begin{itemize}
	\item Many of our recapture pairs are from the same fish (say it was captured 4 times, that's 3 recapture pairs) so the individually-variable $L_{\infty}$ (estimated by the Kirkwood and Somers model and the Sainsbury model) and $K$ (estimated by the Sainsbury model) are going to be off because recaptures of the same fish will have the same $L_\infty$ and $K$, which isn't captured by the model.
	\item We definitely have release length measurement error but unlike the tuna data he uses, we also have recapture length measurement error. Maybe there's a way to make that random variable they add to be the release length measurement error handle both at the same time?
\end{itemize}

\subsection*{Fitting notes from running fishmethods}
When I run growhamp with all the models fit as default and leaving the starting parameters as Michelle had them, there are 22 warnings that come up about NaNs produced, which I think would occur if growth was infinite (like no time between recaptures). There aren't any zero-days in the data set, though, so I'm exploring a bit more to make sure I'm fitting things sensically (without spending tons of time on this and having to write my own stuff).
\begin{itemize}
	\item Fit all models with the same starting values as Michelle used but only one pair of recapture values per fish (chosen randomly for fish caught $>$2 times) - still get some error messages (In log(2 * pi * parms[3]) : NaNs produced)
	\item Fit all models with same starting values as Michelle used but only pairs caught more than one month apart 
	\item Fit all models with same starting values as Michelle used but only pairs caught more than one month apart and only one pair of recapture values per fish
	\item Comparing those three above, definitely get some differences in estimates, both in Linf and K estimates and in error - Kirkwood and Somers model with model error seems particularly off/different than the rest. Also still get some error messages about NaNs.
	\begin{figure}[H] 
    	\centering
    	\includegraphics[width=1.0\textwidth]{\detokenize{../../Growth/Grow_base_results_11Jan2019.pdf}}
    	\caption{The base model, without making changes to data inputs or starting values. }
	\end{figure}
	\begin{figure}[H] 
    	\centering
    	\includegraphics[width=1.0\textwidth]{\detokenize{../../Growth/RunResults_11Jan2019.pdf}}
    	\caption{Run results with different data inputs, comparing model fit. }	
	\end{figure}
	\begin{figure}[H] 
	    \centering
	    \includegraphics[width=1.0\textwidth]{\detokenize{../../Growth/ChangingStartValues_results_11Jan2019.pdf}}
    	\caption{Changing the start values for model error and Linf error.}
	\end{figure}






	\end{itemize}

\section*{Survival - mark-recapture}
\subsection*{Data inputs and model set-up}
We have two ways of tagging a fish - genetic identification from fin-clips and scans of PIT tags. Generally, fish are only clipped if they do not have a tag (which includes not being of taggable size).

Data sets to consider:
\begin{itemize}
	\item all recaptures, genetic + tag, including only certain genetic recaptures (so no site-changing fish, even if lab work doesn't signal error potential)
	\item all recaptures, genetic + tag, including uncertain genetic recaptures too (which can be found by searching "genetic recapture" in the notes section, will give the gen\_id that might match)
	\item just recaptures based on tags, incorporated estimated tag loss (do I really need to do this one?)
\end{itemize}

\subsection*{Cleaning and preparing data}
\begin{itemize}
	\item Adding distance covariate: To account for uneven sampling across years, since we dive slightly different routes and haven't always covered all sections of each site, we include a time-varying covariate that is the minimum distance we surveyed from the anemone where the fish was first caught and marked. We can calculate this easily for all fish after they are first marked, even if we don't catch them, by finding the closest point on the GPS tracks to the anemone where we first caught them.
	\item Adding size covariate: To include sizes for fish at all recapture points, we used the growth curve to estimate sizes for fish not captured or fish with missing capture years. We took the size at which they were previously captured and used the growth curve to project their size at the next sampling time point(s).
\end{itemize}
% # Create several data sets:
% # 1) all recaptures, genetic + tag, but only certain ones
% # 2) all recaptures, genetic + tag, but including uncertain (site-switching) ones
% # 3) just tag recaptures
% # 4) all recaptures, genetic + tag, only certain ones, size estimated for unrecaught fish
% # 5) all recaptures, genetic + tag, including uncertain (site-switching) ones, size estimated for unrecaught fish
% # 6) just tag recaptures, size estimated for unrecaught fish

\subsection*{Tips and notes on MARK}
\begin{itemize}
	\item p.35/62 of Chapter 11 of MARK book (file name: chap11.pdf): Don't standardize individual covariates (that vary with time) - "standardizing them will cause them to no longer relate to one another on the same scale", so things like common slope parameter won't make sense. If you must standardize, do so before putting the values into an encounter histories input file and make sure all the values are included and standardized together so they share a common mean and standard error.
	\item p.49/62 of Chpater 11 of MARK book (chap11.pdf): model averaging with models with individual covariates: "Mechanically, what you would need to do, if doing it by hand, is take the reconstituted values of $\varphi$ for each model, for a given value of the covariate, then average them using the AIC weights as weighting factors (for models without the covariate, the $\beta$ for the covariate is, in fact, 0." - still have the issue of calculating standard errors, though...
\end{itemize}

\subsection*{Mark-recapture models}
\paragraph*{List of models to consider:}
\begin{itemize}
	\item survival constant, recapture constant (eall\_mean.Phi.dot.p.dot)
	\item survival constant, recapture time (eall\_mean.Phi.dot.p.time)
	\item survival time, recapture constant (eall\_mean.Phi.time.p.dot)
	\item survival constant, recapture distance (eall\_mean.Phi.dot.p.dist)
	\item survival size, recapture size (eall\_mean.Phi.size.p.size)
	\item survival size, recapture distance (eall\_mean.Phi.size.p.dist)
	\item survival constant, recapture size + distance (eall\_mean.Phi.dot.p.size.plus.dist)
	\item survival size, recapture size + distance (eall\_mean.Phi.size.p.size.plus.dist)
	\item survival capture stage, recapture size + distance (eall\_mean.Phi.stage.p.size.plus.dist)
	\item survival capture stage, recapture constant (eall\_mean.Phi.stage.p.dot)
	\item survival site, recapture constant (eall\_mean.Phi.site.p.dot)
	\item survival site, recapture size + distance (eall\_mean.Phi.site.p.size.plus.dist)
	% \item survival size+stage, recapture constant
	% item survival size+stage, recapture size+stage
	% \item survival size+stage, recapture size
	% \item survival size+stage, recapture distance
	% \item survival size+stage, recapture stage
	% \item survival stage, recapture constant
	% \item survival size+color, recapture constant
	% \item survival size+color, recapture distance
	% \item survival color, recapture constant
	% \item survival color, recapture distance
\end{itemize}

\paragraph*{Model results and output}
Here are the relative AICs of the models considered, first with all twelve models shown, and second without the models that have site or capture stage as factors.
\begin{figure}[H] 
    	\centering
    	\includegraphics[width=1.0\textwidth]{\detokenize{../Output/MARK_model_comp_28Jan2019.pdf}}
    	\caption{Comparing AICc among models fit with different factors and covariates.}	
	\end{figure}
	\begin{figure}[H] 
	    \centering
	    \includegraphics[width=1.0\textwidth]{\detokenize{../Output/MARK_model_comp_nositeorstage_28Jan2019.pdf}}
    	\caption{Comparing AICc among models fit with different factors and covariates, without site or capture stage considered.}
	\end{figure}
\section*{Estimating recruitment}

\section*{Estimating egg-recruit survival}

\section*{LEP (or LRP)}

\section*{To-do}
\subsection*{Growth}
\subsection*{MARK}
\begin{itemize}
	\item Re-run MARK runs with 0 instead of mean distance in each year for the distances to anems before the first capture of a fish
\end{itemize}
\subsection*{LEP calc}
\subsection*{Other}

\newpage{}

\bibliography{../../../BibTexReferences}
\bibliographystyle{plainnat}

\end{document}

