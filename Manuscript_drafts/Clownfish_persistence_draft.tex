\documentclass[12pt, oneside]{article}   	% use "a msart" instead of "article" for AMSLaTeX format
\usepackage{color}
\usepackage{geometry}                		% See geometry.pdf to learn the layout options. There are lots.
\geometry{letterpaper}                   		% ... or a4paper or a5paper or ... 
%\geometry{landscape}                		% Activate for for rotated page geometry
%\usepackage[parfill]{parskip}    		% Activate to begin paragraphs with an empty line rather than an indent
\usepackage{graphicx}				% Use pdf, png, jpg, or eps§ with pdflatex; use eps in DVI mode
								% TeX will automatically convert eps --> pdf in pdflatex		
\usepackage{amssymb}
\usepackage{amsmath}
\usepackage[compact]{titlesec}
\linespread{1.7}
\usepackage{float}
\usepackage{pdflscape}
%\usepackage{rotating}
\usepackage{soul}
\usepackage{longtable}
%\usepackage{threeparttable}
\usepackage{lineno}
\usepackage[round]{natbib} %round makes parentheses instead of square brackets
\usepackage{url}
%\usepackage{authblk}
\setcounter{secnumdepth}{4}
\titleformat{\paragraph}
{\normalfont\normalsize\bfseries}{\theparagraph}{1em}{}
\titlespacing*{\paragraph}
{0pt}{3.25ex plus 1ex minus .2ex}{1.5ex plus .2ex}
\graphicspath{ {images/} }

\author{}
\author{Allison G.\ Dedrick$^{a, \ast}$ \\
Katrina A.\ Catalano$^a$ \\
Michelle R.\ Stuart$^a$ \\
J.\ Wilson White$^b$ \\
Humberto Montes, Jr.\ $^c$ \\
Malin Pinsky$^a$}

\title{Clownfish metapopulation persistence draft}

\date{} 

\begin{document}
\renewcommand{\topfraction}{0.95}
\maketitle{}

\noindent{} a. Department of Ecology Evolution and Natural Resources, Rutgers University, 14 College Farm Road, New Brunswick, NJ 08901;

\noindent{} b. Oregon State University

\noindent{} c. Visayas State University

\noindent{} $\ast$ Corresponding author; e-mail: agdedrick@gmail.com

(\textit{Author order not yet determined})

%\noindent{} 1. Present address: Rutgers University, 14 College Farm Rd., New Brunswick, NJ, USA 08901.

\bigskip

%\begin{document}

% TO-DOs:
% add correct equation numbers to schematic (and a way of representing the band of sizes that could be recruits?)

%\maketitle{}

%\section*{Abstract}

\linenumbers{}
\modulolinenumbers[3]

\section*{Introduction}

Metapopulation dynamics and persistence depend on the demographic rates at each patch and the connectivity among patches \citep[e.g.][]{hastings_persistence_2006, hanski1998metapopulation}. Many marine species exist in metapopulations, consisting of patch populations connected through dispersal where connectivity affects patch dynamics though individual patches are unlikely to go extinct \citep{kritzer2006marine}. Assessing demographic parameters and levels of connectivity has been particularly challenging for marine species, where much of the mortality and movement happens at larval and juvenile stages when individuals are hard to track and have the potential to travel long distances with ocean currents \citep[e.g.][]{kritzer2006marine, cowen_larval_2009, roughgarden_recruitment_1988}. A need to understand metapopulations for conservation and management, such as siting marine protected areas \citep[e.g.][]{botsford_dependence_2001,white_population_2010}, however, has led to a large body of theory describing how marine metapopulations might persist. % Check Gerber et al. 2005 as a potential reference, here and below

For any population to persist, individuals must on average replace themselves during their lifetimes. In non-spatially structured populations, we use criteria such as the average number of recruiting offspring contributed by each individual during its life (called $R_0$ when the population is age-structured and density-independent) or the growth rate of the population (such as the dominant eigenvalue $\lambda$ of an age-structured Leslie matrix) \citep{caswell_matrix_2001, burgess2014beyond}. To assess replacement, metrics must take into account the demographic processes across the whole life cycle, including how likely individuals are to survive to the next age or stage, their expected fecundity at each stage, and the survival of any offspring produced to recruitment. 

In a spatially-structured population, persistence still requires replacement but in addition to assessing whether the reproductive output and survival of a population is sufficient, we must also consider how the offspring are distributed across space. The spread of offspring is often described using dispersal kernels, probability density functions that give the relative number of settlers with increasing distance from the origin patch \citep[e.g.][]{bode2018estimating}. Connectivity can also be described using a connectivity matrix, where entries give the probability of dispersing from one patch to another, either found by discretizing the dispersal kernel or through direct estimates of pairwise exchange among patches (choose some examples to cite). A long-held paradigm suggested that marine larvae were well-mixed and dispersed far on ocean currents \citep{roughgarden_recruitment_1988}, suggesting widespread connectivity. With the ability to estimate connectivity through natural tags such as otolith microchemistry or genetics and the realization that larvae can alter their dispersal through behavior \citep[e.g.][]{morgan_nearshore_2009}, however, the paradigm is shifting and local persistence of marine populations is seeming more possible. 

When we consider both the demographic processes within patches and the connectivity among them, a metapopulation can persist in two ways: 1) at least one patch can achieve replacement in isolation, or 2) patches receive enough recruitment to achieve replacement through loops of connectivity with other patches in the metapopulation \citep{hastings_persistence_2006, burgess2014beyond}. In the first case (termed self-persistence), enough of the reproductive output produced at a patch is retained at the patch for the patch, and therefore the metapopulation as a whole, to persist. In the second (network persistence), closed loops of connectivity among at least some of the patches - where individuals from one patch settle at another and eventually send offspring back to the first in a future generation - provide the patch with enough recruitment to persist in the network. Though it has been challenging to estimate the parameters necessary to understand how actual metapopulations persist, a large work of theory developed to guide marine protected areas helps predict when each type of persistence is likely to occur \citep[i.e.\ large patches relative to the mean dispersal distance are likely to be self-persistent,][]{botsford_dependence_2001}. % Is this sentence relevant here?

New ways of identifying individuals and determining their origins, such as otolith microchemistry and genetic parentage analysis \citep[e.g.][]{wang2004sibship, wang2014estimation}, however, are making it increasingly possible to estimate both the demographic [ADD EXAMPLE CITATIONS] and the connectivity \citep[e.g.][]{hameed2016inverse, almany2017larval} parameters necessary to assess persistence in real metapopulations. We might expect that populations on isolated islands are the most likely to be self-persistent and \cite{salles_coral_2015} find that the population of orange clownfish (\textit{Amphiprion percula}) at isolated Kimbe Island in Papua New Guinea can likely persist without outside immigration. In contrast, a metapopulation of bicolor damselfish (\textit{Stegastes partitus}) across four isolated islands in the Bahamas does not appear able to persist without outside input \citep{johnson2018integrating}. For populations that exist in patches along a continuous coastline, rather than on separate islands, however, it is still an open question of how patches interact and what is the scale of metapopulation persistence.

We further our understanding of metapopulation dynamics in a network of patches along a coastline through a study of yellowtail clownfish (\textit{Amphiprion clarkii}) in the Philippines. We assess persistence for all patches of habitat within a 30 km stretch of coastline, which exceeds estimates of the dispersal spread for this species \citep{pinsky2010using}, suggesting the network is likely to operate as a contained metapopulation. With seven years of sampling data, we are able to estimate persistence metrics and replacement over the longer term, rather than just capturing a snapshot of the population dynamics. Our annual sampling also enables us to estiamte abundance each year and investigate dynamics through time to compare with the replacement-based persistence metrics. Here, we use a long-term data set from habitat patches on a continuous set of coastline to understand persistence within a local network. % Work on last sentence...

% Testing/validating/demonstrating that theory empirically has been challenging but new technologies in tagging and genetics are making it more feasible. There have been some recent successes. Cite Johnson et al., Salles et al. the recent paper by someone Will works with (Garavelli or something?). Go through (briefly) what they did and found.

\section*{Methods} 

\subsection*{Study system}

We focus on a tropical metapopulation of yellowtail clownfish (\textit{Ampiprion clarkii}) in the Philippines. Like many clownfish species, yellowtail clownfish have a mutualistic relationship with anemones, where small colonies of fish live \citep{buston2003social, fautin1992field}. Yellowtail clownfish are protandrous hermaphrodites and maintain a size-strutured hierarchy; within an anemone, the largest fish is the breeding female, the next largest is the breeding male, and any smaller fish are non-breeding juveniles. The fish on an anemone maintain a strict social and size hierarchy \citep{buston2003social}, with fish moving up in rank to become breeders only after the larger fish have died or left. In the tropical patch reef habitat of the Philippines, yellowtail clownfish spawn once per lunar month from November to May, laying clutches of benthic eggs that the parents protect and tend \citep{ochi1989mating}. Larvae hatch after about six days and spend 7-10 days in the water column before returning to reef habitat to settle in an anemone \citep{fautin1992field}.

% % Other possible life histories in temperate areas or depending on the density of anemones
% \cite{hattori1991life}: different life history pathways found in population of \textit{A. clarkii} in Japan in area where they are the only anemonefish:
% \cite{ochi1989mating}: p.258: could be differences in the mating system/behavior of clownfish depending on how dense the host anemones are in the habitat and how costly it is to move between them; \textit{A. clarkii} in temperate areas off coast of Japan "showed many differences in behavior and morphology in comparison with conspecifics and othre anemonefishes from the primary habitats (coral reefs)."

Clownfish are particularly well-suited to metapopulation studies due to their limited movement as adults and clearly patchy habitat. Once fish have settled, they tend to stay within close proximity of their anemones [XX meters, CITATION]. This makes fish easier to relocate for mark-recapture studies and simiplifies the exchange between patches to only the dispersal during the larval phase. Patches, whether considered to be the reef patch or the anemone territory of the fish, are clearly discrete and easily delineated, which makes determining the spatial structure of the metapopulation clear. Additionally, clear patches make it easier to assess how much of the site has been surveyed. These simplifying characterstics in habitat and fish behavior make clownfish and other similarly territory-based reef fish useful model systems for studies of metapopulation dynamics and persistence \citep[e.g.][]{buston2013marine, salles_coral_2015, johnson2018integrating}.

\paragraph*{Field data collection}

We focus on a set of seventeen patch reef sites spanning approximately 30 km along the western coast of Leyte island in the Philippines (MAP FIGURE). The sites consist of rocky patches of coral reef and are separated by sand flats. Previous work using genetic isolation by distances estimated that yellowtail clowfish larvae have a dispersal spread of about 10 km \citep[range 4-27 km,][]{pinsky2010using}, so our sites were selected to cover and exceed that range. On the north edge, the sites are isolated from nearby habitat with no additional reef habitat for at least 20 km. % Pinsky et al. 2010 paper: (4–27 km with median 11 km)

TO ADD: Figure 1: map of study sites, picture of clownfish
[\textit{(Add figure with map of study sites and a picture of a clownfish(?).}]

Since 2012, members of the team have sampled fish and habitat at most of the sites annually. During sampling, divers using SCUBA and tethered to GPS readers swim the extent of each site. Divers visit each anemone inhabited by yellowtail clownfish, tagging the anemone to be able to track anemones through time. At each anemone, the divers attempt to catch all of the yellowtail clownfish 3.5 cm and larger, taking a non-lethal tail fin-clip from each for use in genetic analysis, measuring the fork length, and noting the tail color (as an indicator of life stage). Starting in the 2015 field season, fish 6.0 cm and larger are tagged with a passive integrated transponder (PIT) tag, unless already tagged. Divers also look for eggs around each anemone and measure and photograph any clutches found. In total, we took fin clips from XX fish and PIT-tagged XX fish across all years and sites combined, with an average of XX fish clipped and XX fish tagged per year.

\paragraph*{Genotyping and parentage analysis}

[\textit{Add in brief overview of genetic methods, with citations to papers with relevant methods and to Katrina's connectivity paper. Include number of fish genotyped.}]

\subsection*{Estimating inputs from empirical data} % Which should come first, this section or the one about persistence metrics?

\begin{figure}[H] % Schematic: NEED TO UPDATE TO HAVE CORRECT EQUATION NUMBERS!!
	\centering
	\includegraphics[width = 1.0\textwidth]{\detokenize{../Plots/Schematic/Schematic.pdf}}
	\caption{Here, we show the data collected for fish at each life stage (life stage boxes are not scaled by length of stage) and how the empirical data fit into the metric calculations. \label{FIG_Schematic}} % MAKE CAPTION BETTER! MAKE SCHEMATIC BETTER TOO!
\end{figure}

\paragraph*{Growth and survival: mark-recapture analyses}

We mark fish through both genetic samples and PIT tags, allowing us to estimate growth and survival through mark-recapture. After matching up recaptures of the same fish identified by genotype or tag, we have a set of encounters of XX marked fish that includes size and stage at each capture time.

For growth, we estimate the parameters of a von Bertalanffy growth curve \citep{fabens1965properties} in the growth increment form relating the length at first capture $L_t$ to the length at a later capture $L_{t+1}$ \citep{hart2009estimating}, where $L_\infty$ is the average asymptotic size across the population and $K$ controls the rate of growth: %check that I actually defined those well... % in the growth increment form (according to Hart and Chute) - need to actually get a copy of the Fabens paper, doesn't seem to be available on Google Scholar

\begin{equation} \label{EQN_VBL} 
\begin{split}
L_{t+1} & = L_t + (L_\infty - L_t)[1 - e^{(-K)}] \\
 & = e^{(-K)}L_t + L_\infty[1 - e^{(-K)}].
\end{split}
\end{equation}

We see from eqn.\ \ref{EQN_VBL} that we would expect the first length $L_t$ and the second length $L_{t+1}$ to be related linearly \citep{hart2009estimating}. From the slope $m = e^{(-K)}$ and y-intercept $b =  L_\infty[1 - e^{(-K)}]$, we can estimate the von Bertalanffy parameters, such that $K = -\ln m$ and $L_\infty = \frac{b}{(1-m)}$. We use the first and second capture lengths for fish that were recaught after a year (within 345 to 385 days) to estimate $L_\infty$ and $K$. We have some fish that were recaptured multiple times so we randomly select only one pair of recaptures from each to use in estimating the parameters, then repeat this process 1000 times to generate a distribution (Fig. \ref{FIG_ParameterInputs}b, \ref{APP_FIG_UncertaintyInputs}d). 

We use the full set of fish encountered multiple times to estimate annual survival $\phi$ and probability of recapture $p_r$ using the mark-recapture program MARK implemented in R \citep{RMark_Laake2013}. We consider several models with year, size, and site effects on the probability of survival and year and size effects on the probability of recapturing a fish on a log-odds scale (see full list in Table \ref{APP_TAB_MARKmodels}). For fish that are not recaptured in particular year, we estimate their size using our growth model (eqn.\ \ref{EQN_VBL}) and the size recorded or estimated in the previous year. Because fish are not well-mixed at our sites and instead stay quite close to their home anemones, we need to swim near an anemone to have a reasonable chance of capturing the fish on it. Therefore, we also consider a distance effect on recapture probability; we use the GPS tracks of divers to estimate the minimum distance between a diver and the anemone for each tagged fish in each sample year and include it as a factor in some of the models.

The best-fit model using model selection with AICc has an effect $b_a$ of fish size on survival, and additive effects $b_1$ and $b_2$ of fish size and shortest distance to anemone on the probability of recapture:

\begin{eqnarray}
\log(\frac{\phi}{1-\phi}) &=& b_\phi + b_a\text{size} \\
\log(\frac{p_r}{1-p_r}) &=& b_{p_r} + b_1\text{size} + b_2d. \label{EQN_Survival}
\end{eqnarray}

\paragraph*{Fecundity}

We use a size-dependent fecundity relationship, determined using photos of egg clutches and females \citep{yawdoszynInPrepfecundity}, where the number of eggs per clutch ($E_c$) is exponentially related to the length in cm of the female ($L$) with size effect $\beta_l = 2.388$, intercept $b = 1.174$, and egg age effect $\beta_e = -0.6083$ dependent on if the eggs are old enough to have visible eyes: 
\begin{equation} % is this the best way of writing this?
\ln(E_c) = \beta_l\ln(L) + \beta_e[\text{eyed}] + b. \label{EQN_Fec}
\end{equation}
To get total annual fecundity $f$, we multiply the number of eggs per clutch by the number of clutches per year $c_e = 11.9$, using the estimate from \cite{holtswarth2017fecundity}.

We only consider reproductive effort once the fish has reached the female stage. Though the size at which a fish transitions to become a breeding female $L_f$ will depend on the size hierarchy in each particular colony [CITATION], we use the average size recaptured fish were first observed as female. 

% For comparison of numbers (from Clownfish\_SP\_Notes:) Moyer (1986): the individual observed for 11 years and thought to live to at least 13 was ”estimated to have contributed about 160,000 propagules in its lifetime,” spent 3 years as a functional male, then outlived 3 mates as a female; fertilized about 45,000 eggs during 3 years as a male, then spawned about 115,000 eggs as a female; fecundity estimate at this site (Miyake-jima in Japan) is 17,500 eggs/yr/female (from Bell (1976))

\paragraph*{Lifetime egg production}
We use an integral projection model (IPM) \citep[e.g.][]{rees2014building} to estimate the total number of eggs produced by one individual (lifetime egg production: LEP), starting at the recruit stage, when individuals have settled and survived to a size we can sample.

In an IPM, the state of the population at time $t$ is described by the distribution of the population over a continuous trait $z$, for which we use size: $n(z,t)$. The total number of individuals in the population at time $t$ is the integral of the size distribution over size from the lower size bound $L$ to the upper size bound $U$: $\int_L^U n(z,t) dz$. The population is projected forward with probability density functions, called the kernel, that describe the survival, growth, and reproductive output of existing individuals into the next time step. 

We initalize the IPM with one recruit-sized individual ($\text{size}_\text{recruit}$): $n(t=0) = n(\text{size}_\text{recruit}, 0)$, then use a kernel with the size-dependent survival and growth functions described above to project forward for 100 time steps. This gives us the size distribution at each time step, which represents the probability that the individual has survived and grown into each of the possible size categories. The probability that the individual is still alive and of any size decreases as the time steps progress; by using a large number of steps, we are able to avoid arbitrarily setting a maximum age and instead let the probabilities become essentially zero. 

We then multiply each size-distribution vector $v_z$ in the matrix by the size-dependent fecundity function described above (eqn.\ \ref{EQN_Fec}) to get the total number of eggs produced at each time step. To get the total number of eggs one individual is likely to produce in its lifetime, we then sum across all time steps in the individual's potential life.  

\begin{equation} % I'm not sure this is quite right...
\text{LEP} = \Sigma_{t=0}^{t=100} \Sigma_{z=L}^{z=U} v_z,t f_z. \label{EQN_LEP}
\end{equation}

\paragraph*{Survival from egg to recruit}

We estimate survival $S_e$ from egg to recruit using the number of recruited offspring we can match back to genotyped parents as surviving individuals from genetically "tagged" eggs in a method similar to that in \cite{johnson2018integrating}. We estimate the number of eggs produced by genotyped parent fish by multiplying the number of genotyped parents ($N_g = 913$) by the expected lifetime egg production of a parent fish $LEP_p$, using LEP calculated starting with an individual of 6 cm. We make the assumption that all recruited offspring originating from the genotyped parents end up in one of the sites we sample and estimate the total number of offspring that survive to recruit $R_t$ by dividing the number of offspring matches we find ($R_m = 90$) by the proportion of our site habitat we sample cumulatively across all sampling years ($P_h = 0.34$) and the probability of capturing a fish if we sample an anemone $P_c$ (see \ref{APP_SEC_ProbHabSampled}, \ref{APP_SEC_ProbR} for details on $P_h$ and $P_c$ estimates, respectively). Our estimated survival from egg to recruit is the number of tagged recruits divided by the number of tagged eggs produced:

\begin{equation}
S_e = \frac{\frac{R_m}{P_h P_c}}{N_g \text{LEP}_p}. \label{EQN_EggRecruitSurv}
\end{equation}

\paragraph*{Defining recruit and census stage} % Not quite sure where to put this section... or if it should even be it's own section

When assessing persistence, it is important to consider mortality and reproduction that occurs across the entire life cycle to determine whether an individual is replacing itself with an individual that reaches the same life stage \citep{burgess2014beyond}. We define a recruit to be a juvenile individual that has settled on the reef within the previous year; lifetime egg production assesses how many offspring an individual recruit is likely to produce in its lifetime from that point forward and egg-recruit survival gives us the fraction of those eggs that will survive to reach the recruit stage. In theory, it should not matter exactly how we define recruit so long we use that definition in our calculations of both egg-recruit survival and LEP. In our system it is straighforward to calculate LEP from any point but it is not possible to change our estimate of egg-recruit survival to allow different definitions of recruit: we do not have enough tagged recruits to reliably estimate survival to different recruit sizes. Instead, we choose the mean size of offspring matched in the parentage study as our best estimate of the size of a recruit ($\text{size}_\text{recruit}$) and test sensitivity to different sizes within the range of sizes that the recruit stage covers (Table \ref{TAB_Params}).

\paragraph*{Probability of dispersal}

We use a distance-based dispersal kernel, estimated in other work using parent-offspring matches from our genetic data \citep{catalanoInPrepconnectivity} using the method described in \cite{bode2018estimating}. The relative dispersal is a function of distance $d$ as measured in kilometers and parameters $\theta$ and $k_d$, which control the shape and scale of the kernel:
\begin{equation}
p(d) = e^k e^{-(e^k d)^\theta}. \label{EQN_DispKernel}
\end{equation}
We use a Laplacian dispersal kernel with shape parameters $\theta = 1$ and scale parameter $k_d = -1.84$ (Fig.\ \ref{FIG_ParameterInputs}a, estimated in \citep{catalanoInPrepconnectivity}).

The dispersal kernel is estimated using fish that have already recruited to a population and survived to be sampled so it gives the relative amount of dispersal given that a fish recruits somewhere, not the probability that a released larva will travel a particular distance. To find the probability of fish dispersing among our sites, we calculate the distance between the middle of each site to the closest and farthest edge of each other site, then use the distances as upper and lower bounds when integrating eqn. \ref{EQN_DispKernel}, which we do numerically. For example, the probability of dispersal from site A to B, where $d_1$ is the distance from the middle of A to the closest edge of B and $d_2$ is the distance from the middle of A to the far edge of B, is:

\begin{equation} % might not need this equation...
p_{A, B}(d) = \int_{d_1}^{d_2} e^k e^{-(e^k d)^\theta}  dd. \label{EQN_integratingDK}
\end{equation}

\subsection*{Persistence metrics}

For a metapopulation to persist, at least one patch needs to achieve replacement, where the number of individuals entering the population balances those lost to mortality or emmigration \citep{burgess2014beyond, hastings_persistence_2006}. In our focal system, adults do not move among patches so we do not need to consider emmigration and only need to assess whether fish produce enough offspring that survive to recruitment to be able to replace themselves and where those offspring travel within the metapopulation. We consider three primary metrics to assess whether and how the population is persistent: 1) lifetime production of recruits, to assess whether the population has enough surviving offspring to achieve replacement 2) self-persistence, to assess whether any individual patches would be able to persist in isolation without any input from other patches, and 3) network persistence, to assess whether the metapopulation is persistent as a connected unit. We explain each metric below in detail. % Reference the section with \ref?

% Should I include some sort of discussion of time series?

\paragraph*{Estimated abundance over time}

[\textit{Add brief section here.}]

\paragraph*{Lifetime production of recruits}

To assess whether individuals at our focal patches produce enough offspring that survive to become recruits themselves, we find the estimated number of recruits an individual recruit will produce over its lifetime (lifetime recruit production: LRP) by multiplying LEP by the estimated survival from egg to recruit $S_e$:
\begin{equation}
\text{LRP} = \text{LEP} * S_e. \label{EQN_LRP}
\end{equation}
If $LRP \geq 1$, the population has the possibility for replacement; indviduals produce enough surviving offspring, before taking into account the probability of dispersal. If $LRP < 1$, the individuals are not replacing themselves and the population cannot persist without input from outside patches.

\paragraph*{Self-persistence}

A patch is able to persist in isolation (self-persistent) if individuals produce enough offspring (LEP) that disperse back to the natal patch and survive to recruitment to be able to replace themselves (LR): $\text{LEP} \times \text{LR} \geq 1$ \citep{burgess2014beyond}. Our dispersal kernel represents the probability that a recruit disperses a distance given that it recruits somewhere, rather than the probability of a larva dispersing and recruiting to a particular patch, which implicitly encompasses mortality from egg to recruitment. We modify the equation to fit our data and include survival from egg to recruit to assess whether a particular patch $i$ is self-persistent: 
\begin{equation}
\begin{split}
SP_i &= \text{LEP} \times \frac{\text{recruits}}{\text{egg}} \times \frac{p_{i,i} \times \text{\# recruits from patch i}}{\frac{\text{recruits}}{\text{egg}} \times \text{\# eggs produced by patch i}} \\ 
%SP_i &= \text{LEP} \times S_e \times \frac{p_{i,i} \times \text{\# recruits from site $i$}}{S_e \times \text{\# eggs produced by patch $i$}} \\
SP_i &= \text{LEP} \times S_e \times p_{i,i}. \label{EQN_SP}
\end{split}
\end{equation}
A patch is self-persistent if $\text{SP} \geq 1$. If at least one patch is self-persistent, the metapopulation as a whole is persistent as well \citep{hastings_persistence_2006, burgess2014beyond}.

\paragraph*{Realized connectivity matrix and network persistence}
We find the probabilities of a recruit dispersing between each set of sites ($p_{i,j}$) by integrating the dispersal kernel (eqn.\ \ref{EQN_DispKernel}) over the distance between each set of sites. We then create a realized connectivity matrix $C$ by multiplying the dispersal probabilities by the expected number of recruits an individual produces: $C_{i,j} = \text{LRP} \times p_{i,j}$ \citep{burgess2014beyond}. The diagonal entries of $C$, where the origin and destination are the same sites, are the values of self-persistence we calculate above. 

Network persistence requires that the largest real eigenvalue of the realized connectivity matrix $\lambda_C$ be greater than $1$: $\text{NP} = \lambda_C > 1$ \citep[e.g.][]{hastings_persistence_2006, white_population_2010, burgess2014beyond}.

\paragraph*{Incorporating uncertainty}
To represent the uncertainty in our estimates of the parameters that go into calculating our persistence metrics, we calculate each metric 1000 times, pulling each parameter from a distribution or range. In our results, we show the range of values of each persistence metric as well as the value with our best estimate of each parameter. 

For the dispersal kernel, we keep the shape parameter $\theta$ constant and pull the scale parameter $k_d$ from a set capturing the 95\% confidence intervals, which was produced during kernel estimation in \cite{catalanoInPrepconnectivity}. To capture uncertainty in the size of a recruit $\text{size}_\text{recruit}$, and therefore the transition of mortality being captured by egg-recruit survival to being captured by LEP, we pull from a uniform distribution over the range of fish sizes (3.5 - 6.0 cm) considered as offspring in the parentage analyses \citep{catalanoInPrepconnectivity}. We include uncertainty in the size of transition to a breeding female $L_F$ by pulling from the set of sizes observed in the data for fish at their first recapture as a female. For the von Bertalanffy growth parameters $L_\infty$ and $K$, we pull from the full set of estimates using different combinations of recapture pairs for fish recaptured more than twice. For uncertainty in adult survival, we pull from a normal distribution generated using the uncertainty estimated in the mark-recapture analysis for both the intercept $b_\phi$ and the size effect $b_a$.

To incorporate uncertainty in egg-recruit survival, we consider uncertainty in both the number of offspring assigned to parents $R_m$ during the parentage analysis and the probability of capturing a fish $P_c$, which affects how the captured assigned offspring are scaled up to account for fish uncaught. For the number of assigned offspring, we generate a set of values of number of assigned offspring using a random binomial, where the number of trials is the number of genotyped offspring (XX) and the probability of success on each trial is the assignment rate XX of offspring from the parentage analysis \citep{catalanoInPrepconnectivity}. To represent uncertainty in the probability of capturing a fish, we pull values from a beta distribution with parameters $\alpha_{P_c}$ and $\beta_{P_c}$, found using the mean and variance of capture probabilities estimated from recapture dives across sites and sampling seasons (details in \ref{APP_SEC_ProbR}). 

\section*{Results}

Our estimated abundance of females at each site over time is relatively constant [\textit{add some sort of actual analysis here}] (Fig.\ \ref{FIG_FthroughTime}), suggesting that our sample populations are stable over time.

\begin{figure}[H] %  abundance trends through time with some sort of time series analysis
	\centering
	\includegraphics[width = 1.0\textwidth]{\detokenize{../Plots/FigureDrafts/Time_series_scaled_F_by_site_with_lines.pdf}}
	\caption{The estimated number of females at each site over the sampling years. The total number of females at each site was estimated by taking the number of females (fish $>$ 5 cm with the yellow pointed tail indicating female) captured at each site in each year and scaling up by the proportion of habitat sampled at that site that season (see \ref{APP_SEC_ProbHabSampled} for details) and by the average probability of capturing a fish (see \ref{APP_SEC_ProbR}). \label{FIG_FthroughTime}}
\end{figure}

From the mark-recapture analysis of tagged and genotyped fish, we estimate mean values of $L_\infty = 10.58 \text{cm}$ (range of estimates 10.39 - 10.75 cm) and $K = 0.928$ (range of estimates 0.854 - 1.025) for the von Bertalanffy growth curve parameters (Fig.\ \ref{FIG_ParameterInputs}b, Table \ref{TAB_Params}). For juvenile and adult (post-recruitment) survival on a log-odds scale, the best-fit model has a coefficient $b_a = 0.74 \pm 0.060$ SE for the effect of size and an intercept $b_\phi = -4.83 \pm 0.340$ SE. These results suggest that larger fish have higher annual survival, which is similar to survival estimates in other clownfish species (check Buston paper). The accompanying best-fit model for log-odds recapture probability has intercept $b_{p_r} = 17.93 \pm 0.858$ SE, size effect $b_1 = -1.816 \pm 0.080$ SE, and effect of diver distance from the anemone $b_2 = -0.171 \pm 0.021$ SE. The negative effect of both size and distance suggest that divers are less likely to recapture larger fish and those at anemones far from areas sampled, with the chance of recapturing an average-sized fish falling below 5\% if a diver stays farther than XX from its home anemone [add the recapture probability plots, like the survival one in Fig.\ \ref{FIG_ParameterInputs}, to the appendix and reference here.] 

We set the transition size to breeding female $L_f$ at 9.32 cm, the mean size of first female capture of recaptured fish (Fig. \ref{FIG_ParameterInputs}d). [\textit{Contextualize these values.}]

% Assume I shouldn't talk about fecundity model results and dispersal kernel results here b/c they are done in other papers? Or should I? But put their parameters below for easy reference?

[\textit{Not sure where to put this table - kind of a methods/results hybrid, or if it should exist, but seems like it might be helpful. Need to clarify somewhere what kind of distributions are going into the uncertainty runs (drawn from data, uniform across a range, 95\% confidence bounds, etc.)}]
\begin{centering}
\begin{longtable}{|p{0.8in}|p{1.2in}|p{1.5in}|p{1in}|p{1.5in}|}
\hline 
\textbf{Parameter} & \textbf{Description} & \textbf{Best estimate} & \textbf{Range in uncertainty runs} & \textbf{Notes} \\ \hline
$k_d$ & scale parameter in dispersal kernel & -1.36 & -2.03 to -0.96 & estimated using methods in \cite{bode2018estimating} in Catalano et al.\ (in prep) \\ \hline
$\theta$ & shape parameter in dispersal kernel & 0.5 & NA & estimated using methods in \cite{bode2018estimating} in Catalano et al.\ (in prep) \\ \hline
$L_\infty$ & average asymptotic size in von Bertalanffy growth curve & 10.58 cm & 10.39 to 10.75 cm &  \\ \hline
$K$ & growth coefficient in von Bertalanffy growth curve &  0.928 & 0.854 to 1.025 & \\ \hline  
$b_\phi$ & intercept for adult survival & -4.83 & $\pm$ 0.340 standard error & \\ \hline
$b_a$ & size effect for adult survival & 0.74 & $\pm$ 0.060 standard error & \\ \hline
$b_{p_r}$ & intercept for recapture probability from mark-recapture analysis & 17.93 & $\pm$ 0.858 standard error & not used in persistence estimates \\ \hline
$b_1$ & size effect for recapture & -1.816 & $\pm$ 0.080 standard error & not used in persistence estimates \\ \hline
$b_2$ & distance effect for recapture & -0.171 & $\pm$ 0.021 standard error & not used in persistence estimates \\ \hline
$\text{size}_\text{recruit}$ & size (cm) of recruited offspring & mean of size of offspring in parentage analysis = 4.4 cm & 3.5 - 6.0 cm & \\ \hline
%$S_e$ & egg-recruit survival & & &  \\ \hline
%$E_c$ & eggs per clutch & depends on female size (eqn.\ \ref{EQN_Fec}) & & relationship from Yawdoszyn et al.\ (in prep) \\ \hline
$b_e$ & coefficient for eyed eggs & -0.608 & & Yawdoszyn et al.\ (in prep) \\ \hline
$b_l$ & size effect in eggs-per-clutch relationship & 2.39 & & Yawdoszyn et al.\ (in prep) \\ \hline
$b$ & intercept in eggs-per-clutch relationship & 1.17 & & Yawdoszyn et al.\ (in prep) \\ \hline
$L_f$ & size at transition to female & 9.32cm & 5.2 - 12.7cm & \\ \hline
$P_c$ & probability of capturing a fish & 0.56 & drawn from beta distribution with parameters $\alpha_{P_c} = 1.44$ and $\beta_{P_c} = 1.13$ & details in \ref{APP_SEC_ProbR} \\ \hline
\caption{}\label{TAB_Params}
\end{longtable}
\end{centering}

\begin{figure}[H] % demographic parameters: survival curve, growth curve, dispersal kernel, transition size to female 
	\centering
	\includegraphics[width = 1.0\textwidth]{\detokenize{../Plots/FigureDrafts/Parameter_inputs.pdf}}
	\caption{Best estimates (solid black line) and range included for uncertainty (gray) for dispersal (a), growth (b), post-recruit survival (c), and size at female transition (d) parameters. \label{FIG_ParameterInputs}}
\end{figure}

Using our best estimates for growth, survival, and fecundity, we calculate a value of LEP for 10876, ranging from XX to XX when we consider uncertainty in the inputs (Fig.\ \ref{FIG_LEP_RperE_LRP}a). The size at recruitment - the cenusus point between egg-recruit survival and LEP - has the most effect on the value of LEP (Fig.\ \ref{APP_FIG_Uncertainty_LEP}), with higher values of LEP the higher the size of recruitment as less mortality is included before reaching reproductive sizes. 

We estimate egg-recruit survival $S_e$ to be 1.82e-05, ranging from XX to XX when we include uncertainty in the number of offspring assigned to parents and the probability of catching a fish (Fig.\ \ref{FIG_LEP_RperE_LRP}b). Uncertainty in the size of transition to breeding female $L_f$ has the largest effect on egg-recruit survival (Fig.\ \ref{APP_FIG_Uncertainty_RperE}); we only consider reproduction from females, to avoid double-counting, so the larger the transition size to female, the fewer tagged eggs we estimate were produced by genotyped parents and the higher egg-recruit survival. % REPHRASE THIS BETTER! THEN TALK ABOUT LRP - what that means for persistence potential.

We estimate lifetime recruit production, the product of LEP and $S_e$, to be 0.20, below the value of 1 necessary for replacement. This suggests that even without considering connectivity, the individuals at our sample populations do not produce enough offspring that survive to recruitment to replace themselves. When we consider uncertainty in our parameter estimates, we do see a few cases where LRP $>$ 1, but the majority are well below the threshold for replacement.

\begin{figure}[H] % LEP, recruits-per-egg, LRP shown with uncertainty, line for best estimate (using mean offspring size as recruit size)
	\centering
	\includegraphics[width = 1.0\textwidth]{\detokenize{../Plots/FigureDrafts/LEP_RperE_LRP.pdf}}
	\caption{Estimates of a) LEP, b) egg-recruit survival, and c) LRP, showing the best estimate (black solid line) and range of estimates considering uncertainty in the inputs. \label{FIG_LEP_RperE_LRP}}
\end{figure}

We do not find any sites with self-persistence values $>$ 1, indicating that the site could persist in isolation. Given that our estimate of LRP does not suggest replacement and only a fraction of that recruitment stays at the natal site, this makes sense. We see the highest values of self-persistence at Haina ($\text{SP} = 0.024$) and Wangag ($\text{SP} = 0.010$), our two widest sites.

\begin{figure}[H] % SP with line for best estimate (using mean offspring size as recruit size)
	\centering
	\includegraphics[width = 1.0\textwidth]{\detokenize{../Plots/FigureDrafts/SP_hists_by_site_noSLSTCP.pdf}}
	\caption{Values of self-persistence at each site, showing the best estimate (black line) and range of estimates considering uncertainty in the input paramters. No site reaches a value of SP $>$ 1, necessary to be self-persistent. \label{FIG_SP}}
\end{figure}

We also do not find evidence of network persistence; the dominant eigenvalue of the realized connectivity matrix $\lambda_c$ is 0.034, well below the value of 1 that indicates network persistence (Fig.\ \ref{FIG_NP_realizedCmat}a). We see that most of the connectivity occurs among the sites in the northern part of our sample area, from Palanas to Caridad Cemetery.

\begin{figure}[H] % NP with line for best estimate (using mean offspring size as recruit size), realized connectivity matrix for best estimate - should re-do without Sitio Lonas, Sitio Tugas, Caridad Proper!
	\centering
	\includegraphics[width = 1.0\textwidth]{\detokenize{../Plots/FigureDrafts/NP_and_connMatrixR.pdf}}
	\caption{a) Network persistence values, showing the best estimate (black solid line) and range of estimates considering uncertainty. b) The realized connectivity matrix $C$, with sites arranged from north (Palanas) to south (Sitio Baybayon). \label{FIG_NP_realizedCmat}}
\end{figure}

Based on our estimates of LRP, SP, and NP, we do not expect that our set of sites is able to persist in isolation as a closed system. To explore what would be required for persistence, we consider a hypothetical scenario in which we consider the system closed and assume that all of the recruits arriving at our sites came from adults at our sites. In this case, we find a value of $\text{LRP} = 1.21$, above the value of 1 necessary for replacement (Fig.\ \ref{FIG_AllOffspringWhatIf}a). When we add in the connectivity, we see a higher value of $\lambda_c$ in our best estimate ($\text{NP} = 0.20$) but still not high enough to indicate network persistence (Fig.\ \ref{FIG_AllOffspringWhatIf}b). We see more of the distribution of estimates above 1, however, suggesting that network persistence is within our range of uncertainty in this case, though not likely. With our site configuration and dispersal kernel estimate, we would need a value of LRP of XX (an egg-recruit survival of XX with our estimated value of LEP or a value of LEP of XX with our estimated value of egg-recruit survival), to $\lambda_c = 1$ and network persistence.

\begin{figure}[H] % Do we see replacement if we consider all recruits when looking at recuits-per-recruit?
	\centering
	\includegraphics[width = 1.0\textwidth]{\detokenize{../Plots/PersistenceMetrics/Whatifs/LEP_R_and_NP_histograms_whatif_all_offspring.pdf}}
	\caption{a) Recruits per recruit when we consider all arriving recruits to have originated from our sites. b) Range of values of NP considering all arriving recruits to be offspring from our sites, with the best estimate in a black solid line. \label{FIG_AllOffspringWhatIf}}
\end{figure}

\section*{Discussion}
Big picture: What do our results mean for persistence in this system and our understanding of metapopulations generally?
\begin{itemize}
	\item So we don't see persistence in our metrics, either self-persistence or network persistence but our abundances don't seem to be changing. Suggests that this is just a portion of a larger metapopulation, rather than a self-contained metapopulation. Maybe it is a sink? Persistent in terms of constant abundance but relies on outside immigration to persist.
	\item How does dispersal spread (estiamted to be within our sites) interact with scale of a self-contained metapopulation? How do we reconcile this in our system, where we don't estimate dispersal that far but don't see network persistence in an area range that spans the estimated spread? (This point might change, depending on mean dispersal distance from the new kernels)
	\item Sensitivity - how would our parameters need to change to see persistence? Egg-recruit survival is a big one. Discuss limitations of how we calculated it (offspring going outside our pops not included - though we might change this), what we see for persistence when estimate recruits/recruits instead. Contextualize this with what other studies have found for these parameters, how reasonable it would be to get better estimates in the field.
\end{itemize}

More detailed discussion of our estimates, limitations, ways to move forward:
\begin{itemize}
	\item Discuss density-dependence: not explicitly accounting for it, included in our egg-recruit survival estimate. But it's these metrics at low abundance, when DD isn't happening, that really matter for persistence. Egg-recruit-survival is probably higher in that case than our estimate of it here (b/c larvae able to settle without being chased off by already-settled recruits). But is it high enough?
	\item Discuss site-specific demographic rates, why we don't esti,ate them in our system, the importance they play in other studies, what we might need to go about resolving them, whether we would expect to see them.
	\item Contextualize our parameter estimates with those from other studies (esp. survival, growth, fecundity).
	% \item Discuss density-dependence: how we might consider it, what not explicitly including it means for our results \\
	% 	\cite{lorenzen2018density}: mortality driven by environmental variability and density-dependence tends to be most important/concentrated in early life stages, density-dependence at that stage can "dampens the variability in abundance induced in earlier life stages and confers compensatory reserve (Myers and Cadigan, 1993; Rose et al. 2001)" \\
	% 	"The relationship betwen stock reproductive output and the number of resulting recruits is described by a stock-recruitment relationship that captures the effects of environmental variability and density-dependence on stock dynamics but does not explicitly model the underlying biological processes." - we just estiamte one number, instead of stock-recruitment curve, in reality, is a function that probably depends on stock size (to capture density dependence) - where do we think our estimate is on that curve? how would including density-dependence shift our estimate? Survival is slope of SR curve (right?) If we're somewhere stable, probably pretty far out on the curve, survival is relatively low (shallow slope), if were to estimate survival at low pop density, would shift left and see higher survivals (how much higher would our survival need to be to see persistence?) \\
	% 	From zebrafish (\textit{Danio rerio}) (Hazlerigg et al. 2012 figure reproduced in Lorenzen and Camp 2018): "Density effects on survival cease at the later juvenile phases" \\
	% 	Often think more about density-dependence in recruitment, tends to predominate, but density-dependent effects on growth can also be important (think more about what that would mean in our system - without the influence of competitors, fish likely to grow either faster - if can just grow to reproductive stage w/out worrying about place in queue - or slower - if not trying to grow extra fast to outpace a smaller competitor in the queue). Complicated in this system, see if can think more about how it might be most likely to play out. Is there a sensitivity test to do here?
	% \item Discuss site-specific demographic rates
\end{itemize}

Broadening back out:
\begin{itemize}
	\item What does this mean for moving forward in understanding metapopulation persistence more broadly? Stability in abundance doesn't mean the population would be able to persist in isolation. Area required seems to be much wider than dispersal kernel spread (particularly if LRP production is right around replacement). Even areas of habitat along a linear coastline seem to be drawing much of their recruitment from a larger surrounding area - even though we see some local retention, maybe broader connectivity is still the story in terms of receiving enough recruitment to persist.
\end{itemize}


% Context for dispersal kernel:
% \begin{itemize}
% 	\item previous research suggests 10km dispersal kernel spread for yellowtail clownfish \citep{pinsky2010using}
% 	\item \cite{hameed2016inverse}: \textit{Petrolisthes cinctipes}, porcelin crab, in N CA had mean dispersal distance of 6.9km (+-sd of 25km) despite 4-6 week PLD
% \end{itemize}

% Context for larval mortality or egg-recruit mortality:
% \begin{itemize}
% 	\item check \cite{white2014planktonic}
% 	\item check estimate in \cite{johnson2018integrating}
% \end{itemize}

% Context for adult mortality
% \begin{itemize}
% 	\item \cite{salles_coral_2015}: check paper, possibly 0.18-0.49 for J, 0.09-0.44 for M, 0.19-0.55 for F (are the ranges for subpops? biannual mortality rates?), orange clownfish (\textit{Ampiphrion percula}) in Kimbe Bay
% 	\item \cite{buston2003social} (or possibly the other 2003 Buston paper...), 12.9\% mortality per year for \textit{Ampiprion percula} in Papua New Guinea
% \end{itemize}


\newpage{}

{\LARGE Appendix}

\appendix

\renewcommand{\theequation}{A\arabic{equation}}
% redefine the command that creates the equation number.
\renewcommand{\thetable}{A\arabic{table}}
\setcounter{equation}{0}  % reset counter 
\setcounter{figure}{0}
\setcounter{table}{0}
\numberwithin{equation}{section}
\numberwithin{figure}{section}

\section{Method details}
\subsection{Proportion of habitat sampled} \label{APP_SEC_ProbHabSampled}
[\textit{Need to add in the details here}]

\newpage{}

\subsection{Probability of capturing a fish, from recapture dives} \label{APP_SEC_ProbR}

We use mark-recapture data from recapture dives done within a sampling season to estimate the probability of capturing a fish. During some of the sampling years (XX), portions of the sites were sampled again XX-XX weeks after the original sampling dives. We assume there is no mortality of tagged fish between the original sampling dives and the recapture dives because they are so close in time and that fish do not change their behavior or reponse to divers, so therefore assume that the probability of recapturing a fish is the same as the probablity of capturing a fish on a sample dive. For each recapture dive, we use GPS tracks of the divers to identify the anemones covered in the recapture dive and the set of PIT-tagged fish encountered on those anemones during the original sampling dives. We estimate the probability of capture $P_c$ as the number of tagged fish caught during the capture dive $m_2$ divided by the total number of fish caught on the recapture dive $n_2$: $P_c = \frac{m_2}{n_2}$. 

We use the mean $P_c$ across all 14 recapture dives, covering XX sites in 3 sampling seasons (2016, 2017, 2018), as our best estimate. Because there are so few recapture dives compared to the number of times we calculate the metrics to show the range of uncertainty, we represent the probability of capture as a distribution, rather than pulling directly from the values calculated for each recapture dive. The distribution of capture probabilities across the 14 dives is quite skewed so we represent it as a beta distribution, using the mean $\mu_{P_c}$ and variance $V_{P_c}$ of the set of 14 values to find the appropriate $\alpha_{P_c}$ and $\beta_{P_c}$ parameters, where 

\begin{eqnarray}
\alpha_{P_c} &=& (\frac{1-\mu_{P_c}}{V_{P_c}} - \frac{1}{\mu_{P_c}}) \mu_{P_c}^2 \\
\beta_{P_c} &=& \alpha_{\mu_{P_c}} \times \frac{1}{\mu_{P_c} - 1}. \label{APP_EQN_ProbCapBetaDistParams}  % Should I cite a source for this?
\end{eqnarray}

% Should I include a table of the recapture probs for each recapture dive, including site and year?

The mean of the individual capture probability values is $\mu_{P_c} = 0.56$, with variance $V_{P_c} = 0.069$, which gives beta distribution parameters $\alpha_{P_c} = 1.44$ and $\beta_{P_c} = 1.13$. We sample 1000 values from the beta distribution, then truncate the sample to only values larger than the lowest value of $P_c$ estimated in an individual dive (0.20), to avoid extremely low values that are sometimes sampled but are unrealistically low. We then sample with replacement from the truncated set to get a vector of values the length of the number of runs.

\newpage{}

\subsection{Full set of MARK models} \label{APP_MARKModels}
We consider the following set of models in MARK [\textit{Need to add in models}]:
\begin{table}
\begin{centering}
\begin{tabular}{|p{2in}|p{2.5in}|p{0.75in}|p{0.75in}|}
\hline 
\textbf{Model} & \textbf{Model description} & \textbf{AICc} & \textbf{dAICc} \\ \hline
& survival size, recapture size+distance & 3348.861 & 0 \\ \hline
& survival size, recapture distance & 3359.998 & -11.1371 \\ \hline
& survival constant, recapture distance & 3383.175 & 34.3141 \\ \hline
& survival constant, recapture size+distance & 3384.959 & 36.0981 \\ \hline
& survival time, recapture constant & 3408.342 & 59.4816 \\ \hline
& survival site, recapture constant & 3440.842 & 91.98112 \\ \hline
& survival site, recapture size+distance & 3440.842 & 91.98112 \\ \hline
& survival constant, recapture time & 3453.609 & 104.74839 \\ \hline
& survival size, recapture size & 3527.710 & 178.84940 \\ \hline
& survival constant, recapture constant & 3570.908 & 222.04690 \\ \hline
\end{tabular}
\end{centering}
\caption{}\label{APP_TAB_MARKmodels}
\end{table}

\section{Uncertainty details}

\subsection{Sensitivity to parameters}

% Range of parameters used as input for uncertainty runs
\begin{figure}[H] % Range of parameter inputs for uncertainty runs
	\centering
	\includegraphics[width = 1.0\textwidth]{\detokenize{../Plots/FigureDrafts/Uncertainty_inputs.pdf}}
	\caption{Range of parameter inputs for uncertainty runs with all uncertainty included: a) $\text{size}_\text{recruit}$, the census size at which fish are considered to have recruited after egg-recruit survival occurs; b) $L_f$, the size at which fish transition from male to female and their reproductive output is included in the estimate of lifetime egg production (LEP); c) $k_d$, the scale parameter in the dispersal kernel; d) the parameters $L\infty$ and $K$ of the von Bertalanffy growth model; e) the intercept $b_\phi$ of the adult size-dependent survival relationship; f) $P_c$, the probability of capturing a fish; g) number of offspring assigned back to parents in the parentage analysis. \label{APP_FIG_UncertaintyInputs}}
\end{figure}

% Relationships among parameters

\begin{figure}[H] % 4 relationships among parameters
	\centering
	\includegraphics[width = 1.0\textwidth]{\detokenize{../Plots/FigureDrafts/Param_metric_relationships.pdf}}
	\caption{Relationships among parameters and metrics. \label{APP_FIG_ParamMetricRelationships}}
\end{figure}

\subsection{Effects of different types of uncertainty on metrics}

\begin{figure}[H] % Uncertainty in LEP
	\centering
	\includegraphics[width = 1.0\textwidth]{\detokenize{../Plots/FigureDrafts/LEP_uncertainty_breakdown.pdf}}
	\caption{The contribution of different sources of uncertainty in LEP. \label{APP_FIG_Uncertainty_LEP}}
\end{figure}

\begin{figure}[H] % Uncertainty in LEP_R
	\centering
	\includegraphics[width = 1.0\textwidth]{\detokenize{../Plots/FigureDrafts/LEP_R_uncertainty_breakdown.pdf}}
	\caption{The contribution of different sources of uncertainty in LRP. \label{APP_FIG_Uncertainty_LEP_R}}
\end{figure}

\begin{figure}[H] % Uncertainty in recruits-per-egg
	\centering
	\includegraphics[width = 1.0\textwidth]{\detokenize{../Plots/FigureDrafts/RperE_uncertainty_breakdown.pdf}}
	\caption{The contribution of different sources of uncertainty in egg-recruit survival. \label{APP_FIG_Uncertainty_RperE}}
\end{figure}

\begin{figure}[H] % Uncertainty in NP
	\centering
	\includegraphics[width = 1.0\textwidth]{\detokenize{../Plots/FigureDrafts/NP_uncertainty_breakdown.pdf}}
	\caption{The contribution of different sources of uncertainty in NP. \label{APP_FIG_Uncertainty_NP}}
\end{figure}



\newpage{}

%\bibliography{../../../BibTexReferences}
\bibliography{BibTexReferences}
\bibliographystyle{plainnat}

\end{document}