\documentclass[12pt, oneside]{article}   	% use "a msart" instead of "article" for AMSLaTeX format
\usepackage{color}
\usepackage{geometry}                		% See geometry.pdf to learn the layout options. There are lots.
\geometry{letterpaper}                   		% ... or a4paper or a5paper or ... 
%\geometry{landscape}                		% Activate for for rotated page geometry
%\usepackage[parfill]{parskip}    		% Activate to begin paragraphs with an empty line rather than an indent
\usepackage{graphicx}				% Use pdf, png, jpg, or eps§ with pdflatex; use eps in DVI mode
								% TeX will automatically convert eps --> pdf in pdflatex		
\usepackage{amssymb}
\usepackage{amsmath}
\usepackage[compact]{titlesec}
\linespread{1.7}
\usepackage{float}
\usepackage{pdflscape}
%\usepackage{rotating}
\usepackage{soul}
\usepackage{longtable}
\usepackage{caption,setspace}
\captionsetup{font={stretch=1.0}}
%\usepackage{threeparttable}
\usepackage{lineno}
\usepackage[round]{natbib} %round makes parentheses instead of square brackets
\usepackage{url}
%\usepackage{authblk}
\setcounter{secnumdepth}{4}
\titleformat{\paragraph}
{\normalfont\normalsize\bfseries}{\theparagraph}{1em}{}
\titlespacing*{\paragraph}
{0pt}{3.25ex plus 1ex minus .2ex}{1.5ex plus .2ex}
\graphicspath{ {images/} }

\begin{document}

\title{Text cut but saved in case reviewers ask about it}

\date{}

\maketitle{}

\section*{Well-edited, cut because of space}
\subsection*{From discussion}
Our sampling on patchy coral reefs was designed for mark-recapture analysis rather than a comprehensive habitat or abundance census, so though we accounted for uneven sampling, we could have missed population declines if the underlying habitat was shifting. We used tagged anemones to account for unvisited patch habitat, but tags with missing anemones are harder to find. If anemones disappeared over time at our patches, we could have overestimated the number of fish and missed population declines indicating lack of persistence even with outside input. These scaling challenges are not unique to our study: few ecological studies are full censuses through time, and marine metapopulations tend to be patchy and heterogeneous \citep[e.g. coral reefs, the intertidal zone, and kelp forests;][]{saenz2011connectivity, johnson2001metapopulation, castorani_connectivity_2015}, where individuals are not well-mixed across space or time. In these cases, carefully considering how sampling interacts with distribution, properly accounting for such uncertainties, and characterizing uncertainty in parameter estimates, is an important part of persistence estimation.

\section*{More random text}
\subsection*{Methods}
% Had been opening paragraph of method, cut to condense
For a population to persist, each individual must on average replace itself \citep[e.g.][]{hastings_persistence_2006, botsford2019population}. In non-spatially structured populations, we use criteria such as the average number of recruiting offspring each individual produces during its life (called $R_0$ when the population is age-structured and density-independent) or the growth rate of the population (such as the dominant eigenvalue $\lambda$ of an age-structured Leslie matrix) \citep{caswell_matrix_2001, burgess2014beyond}. For spatially-structured populations, we must also consider the spatial spread of offspring, often represented through a dispersal kernel or connectivity matrix \citep{burgess2014beyond}. %\citep[e.g.][]{cowen_scaling_2006, buston2011probability, hogan_local_2011, daloia2015patterns}). 

% Moved to supplement
Our equation for SP is a modification of that used in \cite{burgess2014beyond}, which uses LEP to represent offspring produced and uses local retention (the number of surviving recruits that disperse back to the natal patch divided by the number of eggs produced by the natal patch) to capture egg-recruit survival and dispersal combined: $\text{LEP} \times \text{local retention} \geq 1$. We modify this to include egg-recruit survival in the offspring term instead, using LRP in place of LEP. %, to assess whether a particular patch $i$ is self-persistent.

% LEP explanation before moved some details to supplement
We used an integral projection model (IPM) \citep{ellner2016data} with size as the continuous structuring trait $L$ to estimate lifetime egg production on each patch $i$ ($\text{LEP}_i$). We initialized the IPM with one recruit-sized individual (recruit defined in SI \ref{APP_SEC_METHODS_Recruit_def}) at the initial annual time step ($t=0$), then projected forward for 100 years. We used the size- and site-dependent survival (eqn.\ \ref{APP_EQN_Survival}) and growth (eqn.\ \ref{EQN_VBL}) functions as the probability density functions in the kernel to project the individual into the next time step. The size distribution ($v_L$) at each time step represents the probability that the individual has survived and grown into each of the possible size categories, ranging from a minimum of $L_s=0$ cm to a maximum of $U_s=15$ cm divided into 100 equal size bins. %The probability that the individual is still alive and of any size decreases as the time steps progress; by using a large number of steps, we are able to avoid arbitrarily setting a maximum age and instead let the probabilities become essentially zero. %($\text{size}_\text{recruit}$) 

We then multiplied the size-distribution $v_{L,t}$ at each time by the size-dependent fecundity $f_L$ (eqn.\ \ref{EQN_Fec}) to get the total number of eggs produced at each time step. Integrating across time and size gave the total number of eggs one recruit produced in its lifetime (details in \ref{APP_SEC_METHODS_LEP}, uncertainty details in SI \ref{APP_SEC_Uncertainty}): % We only considered reproductive effort once the fish was female and used the average size of first female observation for recaptured fish as the transition size ($L_f = 9.32$ cm). Moved this up to fecundity section.

% From intro
% MOVE TO DISCUSSION PER WILL'S COMMENT ON GOOGLE DOC! % The number of studies estimating demographic rates and connectivity in marine metapopulations is growing \citep{carson2011evaluating, salles_coral_2015, johnson2018integrating, garavelli2018population}, but most use data from one or a few years. Longer data sets enable better estimates of long-term average rates, rather than assuming the demographic and dispersal rates from a particular year or two are representative. Long data sets are also useful for explicitly considering uncertainty, both to assess how well we understand persistence for a population and to assess which parameters contribute most to our uncertainty. Finally, sampling over many years provides abundance trends to compare with persistence metrics.

% Characterizing and understanding variability is important as it can drive dynamics, even allowing persistence when average rates would not (SEBASTIAN CITATIONS), but using data across several years to get a sense of the range of rates and an average is a first step. Having multiple years of data can provide a range of estimates, rather than one point value, and help us understand which parameters contribute most to uncertainty in persistence. 

\subsection*{Discussion}

% 32\% annual survival for larger size classes of Thalassoma bifasciatum (bluehead wrasse), calculated from monthly survivorship, warner1988population
% 28\% annual survival (calculated from mean montly survival across sites, 0.90) for Stegastes partitus (bicolour damselfish)
% 10-32\% annual survival for Stegastes partitus (bicolour damselfish), Figueira 2008
% 50-70\% survival for juvenile Dascyllus aruanus for one year, about 50\% survival for juvnile Pmoacentrus amboinesis over one year
%%%%%%%% NEED TO ADD IN SURVIVALS AND KIMBE BAY LRP equivalent! survival for a 6.0cm fish at Tomakin Dako is 38%, 27% at Elementary School, 10% at N. Magbangon (lowest); for 9 cm fish 15% at N. Magbangon, 37% at Elementary School, 53% at Tomakin Dako

% Longer density dependence discussion paragraph
% Density dependence also presents a sampling challenge. Persistence criteria \citep{hastings_persistence_2006, burgess2014beyond} ask whether a population at low abundance can grow and recover rather than going extinct. Density dependence is often ignored at low abundances \citep{botsford2019population} so is not explicitly considered in persistence metrics. In real populations, however, it can be challenging to estimate density-independent demographic rates because density dependence is occurring in the population as it is sampled during dispersal \citep{nowicki2011evidence} and reproduction \citep{rodenhouse2003multiple}. In yellowtail anemonefish, density dependence is likely most important immediately post-settlement, as it is for many species, including corals, trees, and butterflies \citep{vermeij2008density, harms2000pervasive, nowicki2009relative}. However, density dependence could continue to be important throughout the life history due to social hierarchies in anemonefish colonies \citep{buston2011determinants}. To avoid competition within the colony, fish in the pre-reproductive queue may have lower growth and survival than fish alone on an anemone \citep{buston2003social, buston2003forcible}, suggesting higher growth and survival, and therefore LRP, in the absence of density dependence. Our calculations of self-persistence in this paper did not account for post-settlement density dependence, which would be an interesting area of further research.

%%%%% OLD CONCLUDING PARAGRAPH BEFORE MALIN/WILL EARLY AUGUST EDITS
% Understanding persistence is critical for the management of spatial populations, such as siting marine protected areas \citep{kaplan_model-based_2009}, assessing habitat fragmentation risks \citep{smith2002population, fahrig2001much} and conserving species in the face of climate change \citep{coleman2017anticipating, fuller2015persistence}. Though models and theory provide us with expectations, we are only recently beginning to be able to tackle these questions of persistence empirically in model systems such as anemonefish and other sedentary tropical reef fish \citep{salles_coral_2015,johnson2018integrating}. With parentage analyses now being extended to temperate marine species \citep{baetscher2019dispersal} and a better understanding of how biophysical models compare to larval dispersal patterns \citep{bode2019validation}, we are beginning to move beyond model species and investigate persistence in harvested and spatially-managed systems \citep{garavelli2018population}. Our study shows the importance of long-term sampling and careful consideration of the demographic and sampling processes that affect persistence calculations in order to determine persistence mechanisms and assess persistence state to understand marine population dynamics in empirical systems.


%\bibliography{../../../BibTexReferences}
\bibliography{BibTexReferences}
\bibliographystyle{plainnat}

\end{document}